\documentclass[a4paper,11pt]{article}

% load packages
\usepackage[utf8]{inputenc}
\usepackage{amsmath}
\usepackage{amssymb}
\usepackage{array}
\usepackage{mathtools}
\usepackage{bbm}

% define own colors and use colored text
\usepackage{xcolor}

\usepackage{tikz}
\usetikzlibrary{arrows}
\usetikzlibrary{positioning}
\usetikzlibrary{shapes}
\usetikzlibrary{fit}
\usetikzlibrary{backgrounds}

%%% standard additional commands
%epsilon
\newcommand*{\oldepsilon}{\epsilon}%
\renewcommand*{\epsilon}{\varepsilon}%

% Number Sets
\newcommand{\R}{\mathbb{R}}
\newcommand{\Z}{\mathbb{Z}}
\newcommand{\Q}{\mathbb{Q}}
\newcommand{\N}{\mathbb{N}}
\newcommand{\C}{\mathbb{C}}
\newcommand{\1}{\mathbf{1}}

% Statistics
\newcommand{\Var}{\mathbb{V}}
\newcommand{\Exp}{\mathbb{E}}

% Matrices
\newcommand{\X}{\mathbf{X}}
\newcommand{\Y}{\mathbf{Y}}

% operators
\DeclareMathOperator{\cov}{cov}
\DeclareMathOperator{\di}{d}



\begin{document}

\section{Description}
The Dunning-Kruger effect is stated in terms of skill and overconfidence to show that measurement error can cause significant bias in the relationship between performance and overestimation.
Instrumental variable methods can be used to correct for this bias.

\section{Ontology}
The constructs used are
\begin{itemize}
 \item performance
 \item performance estimation
 \item skill
 \item confidence
 \item measurement error
\end{itemize}

\section{Relationships}

\section{Mathematical Formulation}

\section{(Statistical) Models}

\section{Open Questions / Ambiguities}


\end{document}
