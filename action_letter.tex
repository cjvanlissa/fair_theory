% Taken from https://github.com/mschroen/review_response_letter
% GNU General Public License v3.0

\documentclass[draft]{article}

\usepackage[includeheadfoot,top=20mm, bottom=20mm, footskip=2.5cm]{geometry}

% Typography
\usepackage[T1]{fontenc}
\usepackage{times}
%\usepackage{mathptmx} % math also in times font
\usepackage{amssymb,amsmath}
\usepackage{microtype}
\usepackage[utf8]{inputenc}

% Misc
\usepackage{graphicx}
\usepackage[hidelinks]{hyperref} %textopdfstring from pandoc
\usepackage{soul} % Highlight using \hl{}

% Table

\usepackage{adjustbox} % center large tables across textwidth by surrounding tabular with \begin{adjustbox}{center}
\renewcommand{\arraystretch}{1.5} % enlarge spacing between rows
\usepackage{caption}
\captionsetup[table]{skip=10pt} % enlarge spacing between caption and table

% Section styles

\usepackage{titlesec}
\titleformat{\section}{\normalfont\large}{\makebox[0pt][r]{\bf \thesection.\hspace{4mm}}}{0em}{\bfseries}
\titleformat{\subsection}{\normalfont}{\makebox[0pt][r]{\bf [RC \thesubsection.]\hspace{4mm}}}{0em}{\bfseries}
\titlespacing{\subsection}{0em}{1em}{-0.3em} % left before after

% Paragraph styles

\setlength{\parskip}{0.6\baselineskip}%
\setlength{\parindent}{0pt}%

% Quotation styles
\usepackage[usenames,dvipsnames]{xcolor}

% \usepackage{xcolor}
\newenvironment{quotebox}
  {\def\FrameCommand{
	\fboxsep=0.6em % box to text padding
	\fcolorbox{black!25!white}{white}}%
	% the "2" can be changed to make the box smaller
    \MakeFramed {\advance\hsize-2\width \FrameRestore}
    \begin{minipage}{\linewidth}
  }
  {\end{minipage}\endMakeFramed}

\usepackage{framed}
\let\oldquote=\quote
\let\endoldquote=\endquote
\renewenvironment{quote}{\begin{quotebox}\advance\leftmargini -2.4em\begin{oldquote}}{\end{oldquote}\end{quotebox}}


% Table styles

\let\oldtabular=\tabular
\let\endoldtabular=\endtabular
\renewenvironment{tabular}[1]{\begin{adjustbox}{center}\begin{oldtabular}{#1}}{\end{oldtabular}\end{adjustbox}}


% Shortcuts

%% Let textbf be both, bold and italic
%\DeclareTextFontCommand{\textbf}{\bfseries\em}

%% Add RC and AR to the left of a paragraph
%\def\RC{\makebox[0pt][r]{\bf RC:\hspace{4mm}}}
%\def\AR{\makebox[0pt][r]{AR:\hspace{4mm}}}

%% Define that \RC and \AR should start and format the whole paragraph
\usepackage{suffix}
\long\def\RC#1\par{\makebox[0pt][r]{\bf RC:\hspace{4mm}}{\bf #1}\par\makebox[0pt][r]{AR:\hspace{10pt}}} %\RC
\WithSuffix\long\def\RC*#1\par{{\bf #1}\par} %\RC*
% \long\def\AR#1\par{\makebox[0pt][r]{AR:\hspace{10pt}}#1\par} %\AR
\WithSuffix\long\def\AR*#1\par{#1\par} %\AR*


\newlength{\emptysubsection}
\setlength{\emptysubsection}{0.9em}
\addtolength{\emptysubsection}{\parskip}

\newenvironment{reviewer}{%
\subsection{}
\vspace{-\emptysubsection}
\bfseries
}{}



%%%
%DIF PREAMBLE EXTENSION ADDED BY LATEXDIFF
%DIF UNDERLINE PREAMBLE %DIF PREAMBLE
\RequirePackage[normalem]{ulem} %DIF PREAMBLE
\RequirePackage{color} %DIF PREAMBLE
\definecolor{offred}{rgb}{0.867, 0.153, 0.153} %DIF PREAMBLE
\definecolor{offblue}{rgb}{0.0705882352941176, 0.168627450980392, 0.717647058823529} %DIF PREAMBLE
\providecommand{\DIFdel}[1]{{\protect\color{offred}\sout{#1}}} %DIF PREAMBLE
\providecommand{\DIFadd}[1]{{\protect\color{offblue}\uwave{#1}}} %DIF PREAMBLE
%DIF SAFE PREAMBLE %DIF PREAMBLE
\providecommand{\DIFaddbegin}{} %DIF PREAMBLE
\providecommand{\DIFaddend}{} %DIF PREAMBLE
\providecommand{\DIFdelbegin}{} %DIF PREAMBLE
\providecommand{\DIFdelend}{} %DIF PREAMBLE
%DIF FLOATSAFE PREAMBLE %DIF PREAMBLE
\providecommand{\DIFaddFL}[1]{\DIFadd{#1}} %DIF PREAMBLE
\providecommand{\DIFdelFL}[1]{\DIFdel{#1}} %DIF PREAMBLE
\providecommand{\DIFaddbeginFL}{} %DIF PREAMBLE
\providecommand{\DIFaddendFL}{} %DIF PREAMBLE
\providecommand{\DIFdelbeginFL}{} %DIF PREAMBLE
\providecommand{\DIFdelendFL}{} %DIF PREAMBLE
%DIF END PREAMBLE EXTENSION ADDED BY LATEXDIFF

% Fix pandoc related tight-list error
\providecommand{\tightlist}{%
  \setlength{\itemsep}{0pt}\setlength{\parskip}{0pt}}

% Add task difficulty and assignment commands from https://github.com/cdc08x/letter-2-reviewers-LaTeX-template
% \usepackage[usenames,dvipsnames]{xcolor}
\usepackage{ifdraft}

\newcommand{\TaskEstimationBox}[2]{%
\ifoptiondraft{\parbox{1.0\linewidth}{\hfill \hfill {\colorbox{#2}{\color{White} \textbf{#1}}}}}%
{}%
}
%
\def\WorkInProgress {\TaskEstimationBox{Work in progress}{Cyan}}
\def\AlmostDone {\TaskEstimationBox{Almost there}{NavyBlue}}
\def\Done {\TaskEstimationBox{Done}{Blue}}
%
\def\NotEstimated {\TaskEstimationBox{Effort not estimated}{Gray}}
\def\Easy {\TaskEstimationBox{Feasible}{ForestGreen}}
\def\Medium {\TaskEstimationBox{Medium effort}{Orange}}
\def\TimeConsuming {\TaskEstimationBox{Time-consuming}{Bittersweet}}
\def\Hard {\TaskEstimationBox{Infeasible}{Black}}
%
\newcommand{\Assignment}[1]{
%
\ifoptiondraft{%
\vspace{.25\baselineskip} \parbox{1.0\linewidth}{\hfill \hfill \vspace{.25\baselineskip} \normalfont{Assignment:} \normalfont{\textbf{#1}}}%
}{}%
}


  % \usepackage[none]{hyphenat}
  \usepackage{hyphenat}
  \usepackage{threeparttable}
  \usepackage{makecell}
  \usepackage{booktabs}
  \usepackage{longtable}
  \usepackage{mdframed}
  \usepackage[draft]{todonotes}
  \overfullrule=0pt

  % set up reviewer comments
  \newcounter{counter}[section]
  \renewcommand{\thecounter}{\arabic{counter}}

  \newenvironment{reviewercomment}[1][]{%
   \refstepcounter{counter}%
   \noindent\textsc{Comment}~\thecounter
   }%
   {\par}%

  \newcommand{\rcomm}[1]{\begin{reviewercomment}\label{#1}\end{reviewercomment}}
  \newcommand{\reviewerid}[2]{\RC{\underline{\rcomm{#1}}\vspace{-0.1cm}{#2}}}
  %replace quote with mdframed
  \renewenvironment{quote}
    {% \begin{quote}
     \begin{mdframed}%
    }{% \end{quote}
     \end{mdframed}%
    }

  %use hyphens sparingly
  \hyphenpenalty=50

  %commands for linking to colours
  \newcommand{\comment}[1]{\underline{\hyperref[#1]{comment }\ref{#1}}}
  \newcommand{\Comment}[1]{\underline{\hyperref[#1]{Comment }\ref{#1}}}

  %link colours


% definitions for citeproc citations
\NewDocumentCommand\citeproctext{}{}
\NewDocumentCommand\citeproc{mm}{%
  \begingroup\def\citeproctext{#2}\cite{#1}\endgroup}
\makeatletter
 % allow citations to break across lines
 \let\@cite@ofmt\@firstofone
 % avoid brackets around text for \cite:
 \def\@biblabel#1{}
 \def\@cite#1#2{{#1\if@tempswa , #2\fi}}
\makeatother
\newlength{\cslhangindent}
\setlength{\cslhangindent}{1.5em}
\newlength{\csllabelwidth}
\setlength{\csllabelwidth}{3em}
\newenvironment{CSLReferences}[2] % #1 hanging-indent, #2 entry-spacing
 {\begin{list}{}{%
  \setlength{\itemindent}{0pt}
  \setlength{\leftmargin}{0pt}
  \setlength{\parsep}{0pt}
  % turn on hanging indent if param 1 is 1
  \ifodd #1
   \setlength{\leftmargin}{\cslhangindent}
   \setlength{\itemindent}{-1\cslhangindent}
  \fi
  % set entry spacing
  \setlength{\itemsep}{#2\baselineskip}}}
 {\end{list}}

\begin{document}

{\Large\bf Author response to reviews of}\\[1em]

{\Large To be FAIR: Theory Specification Needs an Update}\\[1em]
{Caspar van Lissa on behalf of co-authors}\\
{submitted to \it Perspectives on Psychological Science }\\
\hrule

\hfill {\bfseries [RC]\hspace{0.5em}} \textbf{Reviewer comment}%
%\(\qquad\) AR:\hspace{0.5em} Author response%
\(\qquad \fcolorbox{black!25!white}{white}{\hspace{0.25em}Manuscript text\hspace{0.25em}}\)

\vspace{2em}

Dear Dr Vrieze,

Thank you for considering our manuscript for publication at \emph{Perspectives on Psychological Science}.
We appreciate the feedback that you and the reviewers have provided.
In the following itemised list we respond to each comment point by point.

\subsection{Editor's Comments}\label{editors-comments}

\reviewerid{unnamed-chunk-1}{


Thank you for submitting Manuscript ID PPS-25-140 entitled "To be FAIR: Theory Specification Needs an Update" to Perspectives on Psychological Science. It now has been reviewed.  The comments of the reviewer(s) are included at the bottom of this letter.

The reviewer(s) have recommended publication, but also suggest some revisions to your manuscript. I have some further comments, which I hope also will be helpful to you in making a revision. Therefore, I invite you to respond to the reviewer(s)' and my comments and to revise your manuscript.

[x] Headings should not be numbered.
[] Tables and Figures should be uploaded separately and not positioned in the text (although it is fine to indicate where they should be inserted).
[] Figures should be uploaded in their native format (the program they were created in) at 300 dpi. Acceptable figure file formats include MS Office files (DOC, PPT, XLS), Adobe Illustrator (AI), JPG, EPS, and PDF.
[x] References should be done with hanging indents using APA style. 
[] If you want to add the doi that is good.  But do not put in the search criteria for retrieving a document that has been published in a book or journal. WHAT??
}

Upload tables separately
Upload figures separately

\reviewerid{unnamed-chunk-2}{


a) You have a balancing act to strike between providing technical tools (e.g., version control workflows) that can assist scientists in implementing your proposals on the one hand, against a readership and a scientific discipline that likely largely cannot effectively implement such tools. I will give a simple example. The very first thing I read after the title page was "This is a preprint paper, generated from Git Commit \# 7b710ba5." I wager that most readers of Perspectives do not know what a git commit is, at least not in any meaningful way. This is a minor example, but illustrates the larger point. Reviewer 2 (Turkheimer) makes this point in his review.
}

This does not directly ask for a change, but summarize relevant changes here.

\reviewerid{unnamed-chunk-3}{


b) My interpretation of this manuscript is that it is proposing policies, procedures, and documentation as an important path forward to improve psychology and related soft sciences. One reason I find such proposals interesting is because I don't see them in my other area of expertise, human genetics. No doubt strong conventions develop in genetics, but they seem to develop because scientists writ large working in those fields recognize that those conventions (e.g., publicly available genomic maps organized in a certain way; analytical workflows based on accepted genetic theories) will immediately benefit their own area of research by making new avenues of work feasible, or more efficient. The payoff is obvious. They are "no-brainers" that organically arise. Some conventions in psychology, like registration or even data sharing, do not seem to arise or take hold in quite the same way, and instead rely on top-down requirements by grant funding agencies, journals, etc. 

I can definitely see the advantage to policies, procedures, and documentation, if the goal is to identify fraudulent work, because it creates a paper trail to audit. I am more skeptical that these things are crucial to scientific advance in psychology, and there are no empirical results in this paper to the contrary. I believe this echoes Reviewer 3's suggestion to take seriously in your discussion, and I would suggest elsewhere in the paper, the limitations of policies, procedures, and documentation, versus other aspects of the scientific enterprise including culture, norms, and perhaps even the tractability of the subject matter, as in my human genetics example. 
}

Make comparison to preregistration; that was first proposed and only later evaluated.
Mention how few of preregistered hypotheses are true, and yet how theory is rarely updated? Thus, falsificationism is not working in practice.

\subsection{Reviewer: 1}\label{reviewer-1}

\reviewerid{unnamed-chunk-4}{


In this manuscript, the authors introduce a framework for FAIR theory specification while outlining what makes a theory FAIR, which is accompanied by practical guidance on how to implement this in one’s own work. In my opinion, the paper is relevant for the journal and presents research that is significant for the field of psychology but also adjacent fields that rely on theories similar in nature to ours. FAIR theory is a promising idea that could significantly advance psychological research, especially in terms of transparency and replicability.
}

\reviewerid{unnamed-chunk-5}{


The abstract provides a (mostly) comprehensive overview of the paper that can be easily followed and understood. I personally felt that the introduction of the theorytools R-package was an important aspect of the paper and should thus briefly be mentioned in the abstract.
}

\reviewerid{unnamed-chunk-6}{


In the abstract, the authors mention the discussion of “FAIR theories’ potential impact in terms of reducing research waste,” which I couldn’t really find at any point in the manuscript.
}

\reviewerid{unnamed-chunk-7}{


Lines 48–50: I would have benefited from a few more words on the FAIR guiding principles (what they are, where/how they originated, and how they are used).
}

\reviewerid{unnamed-chunk-8}{


Line 73 onwards: The example of SDT nicely illustrates the authors’ point here!
}

\reviewerid{unnamed-chunk-9}{


Line 101: In Figure 1a, it would be helpful to see the two phases (currently, those are not clear to me). Also, even though Figure 1c is addressed at a much later point in the manuscript, it would be helpful to briefly mention it here as well, so it doesn’t seem forgotten.
}

\reviewerid{unnamed-chunk-10}{


Line 102: The corresponding year should be added to “Wagenmakers and colleagues.”
}

\reviewerid{unnamed-chunk-11}{


Line 147: In the case of “Lamprecht and colleagues,” the year should also be added.
}

\reviewerid{unnamed-chunk-12}{


In general, I’d move this section up, as the word “theory” has been used a lot already at this point, and readers might benefit from an earlier explanation/definition (for example, directly after the section “The Need for FAIR Theory”).
}

\reviewerid{unnamed-chunk-13}{


Line 163: Following APA 7 standards, no “cf.” is added. In the case of three or more authors, it is shortened to “et al.”
}

\reviewerid{unnamed-chunk-14}{


Line 207: It could be helpful to add a very brief description of the phonological loop in case some readers are not familiar with it and its parameters.
}

\reviewerid{unnamed-chunk-15}{


Line 350: I think that this was supposed to be displayed as a link: <www.theorymaps.org>
}

\reviewerid{unnamed-chunk-16}{


Line 390: In the case of a direct quotation, a page number should be added if possible.
}

\reviewerid{unnamed-chunk-17}{


This is a very nice section; I really enjoyed reading this more practical angle!
}

\reviewerid{unnamed-chunk-18}{


Line 688: I’d add the year for Van Lissa’s specification.
}

\reviewerid{unnamed-chunk-19}{


Line 776 onwards: The proposed system of modular publishing seems to have been around for quite some time (since 1998 at least) but is not widely used, which could briefly be addressed as a limitation.
}

\reviewerid{unnamed-chunk-20}{


Line 800: To make the strengths understandable to all readers, I’d encourage the authors to briefly explain the “open empirical cycle.”
Overall, this is a very “strong” section indeed!
}

\reviewerid{unnamed-chunk-21}{


In my opinion, it is very important that the learning curve is mentioned; even though I am a big fan of the proposed framework in general, I do feel like it would take some practice and getting used to.
}

\reviewerid{unnamed-chunk-22}{


Line 852: I’d add a reference for the jingle-jangle fallacy.
}

\reviewerid{unnamed-chunk-23}{


The affiliations could be changed to follow a common system.
}

We tried to make this suggested change, although there is still some heterogeneity because our different institutions have different preferred / official ways to write the affiliation.

\reviewerid{unnamed-chunk-24}{


I do understand why the words Findable, Accessible, Interoperable, and Reusable are always written with a majuscule, but it did confuse me a bit during reading at times.
}

Since all of these words have a colloquial meaning in everyday language, as well as a more restrictive operationalization in the literature on the FAIR principles, we think its important to signal the distinction between the formal and colloquial use of the terms. We now clarify this in a footnote, the first time the capitalized terms are introduced:

\reviewerid{unnamed-chunk-25}{


The references should use sentence case following APA 7 (e.g., Back to Basics: The Importance of Conceptual Clarification in Psychological Science → Back to basics: The importance of conceptual clarification in psychological science).
}

\reviewerid{unnamed-chunk-26}{


I’ve mentioned these two already for specific examples, but I’d check them in the whole manuscript again: In the case of more than two authors, use of et al. and the mention of page numbers for direct quotations.
}

\reviewerid{unnamed-chunk-27}{


Two of the links did not work for me: Line 490 (FAIR metadata example), Line 606 (example of metadata).
}

\subsection{Reviewer: 2}\label{reviewer-2}

\reviewerid{unnamed-chunk-28}{


Editors generally prefer that reviewers not preface their reviews with declarations of their lack of expertise in the relevant subjects, but I am going to allow myself to do so here, because I already said as much to the editor when I was asked to review the paper. I have no particular expertise in meta-science; I have never (I am somewhat embarrassed to admit) so much as pre-registered an analysis. I am 71, and open science procedures are the blinking 12:00 on my VCR.

I am here as a practitioner of theory, not as a meta-theorist. I have written a few "theoretical" papers over the years, of different kinds that I will get to in a minute.  I want to consider how adopting a FAIR framework would change, improve, or inhibit what I do.

I can think of two kinds of theoretical papers I have written, or perhaps better, two theoretical research programs I have participated in.  One of them is closely tied to empirical results. I am thinking of the Scarr-Rowe hypothesis, something that was developed by Sandra Scarr and David Rowe, and then turned into something resembling a theory by our 2003 paper.   Since then it has been much investigated and argued about.

In this case, I think the FAIR framework would be very helpful. The S-R hypothesis makes a number of empirical claims with respect to theory. These claims have changed in response to empirical results and theoretical disputes over the years.  I have often been frustrated by the difficulty of maintaining meaningful boundaries on the theory and the predictions it makes.
}

\reviewerid{unnamed-chunk-29}{


Here are two examples. In the months after our paper was published, someone (Nagoshi, C. T., \& Johnson, R. C. (2005). Socioeconomic status does not moderate the familiality of cognitive abilities in the Hawaii Family Study of Cognition. Journal of Biosocial Science, 37(6), 773-781.) shot out a "failure to replicate." But it wasn't-- their analysis studied parent-child correlations, not twins, and the S-R hypothesis makes no predictions about them. But despite 20 years of trying to make this clear in the published literature, the study is still cited. An even worse example happened recently, when a group of "race scientists" (Pesta, B. J., Kirkegaard, E. O., te Nijenhuis, J., Lasker, J., \& Fuerst, J. G. (2020). Racial and ethnic group differences in the heritability of intelligence: A systematic review and meta-analysis. Intelligence, 78, 101408.) used the S-R hypothesis as a platform for a bunch of unrelated claims about race differences.
}

\reviewerid{unnamed-chunk-30}{


To be clear, it cuts both ways. It has turned out over the years that the S-R interaction does not occur in Europe, and I have revised the theory to fit the new facts, but I have done so on a strictly ad hoc basis. It would be easy to suggest that I am adding epicycles to it, and having a formalized structure for the theoretic revisions would help keep me honest.  Similarly, when conducting new analyses, I am often confronted with investigator degrees of freedom (should we use parental education or income as an SES indicator) that are very difficult to resolve in the privacy of the lab.  Having a structured platform for doing so would help. (This sounds like pre-registration. I don't think there is a bright line between FAIR theorizing and preregistering.)
}

Discuss relationship betwene FAIR theory and prereg? That sounds like a job for Aaron!

\reviewerid{unnamed-chunk-31}{


The second kind of theorizing I have done is more philosophical, along the lines of my "Three Laws" paper and related work. Here, I think applying the FAIR framework would be a little more problematic. At times, the current manuscript sounds as though it is trying to make scientific theory machine readable, an idea that has some good points that the authors make clear, but would also eliminate a lot of interesting papers that have been written by human beings from a subjective point of view. What would happen to Meehl's great Sir Karl Sir Ronald paper within FAIR? How much of the importance of that paper would remain if you took Meehl's voice out of it? Is all narrative philosophy "theorizing" that could be augmented or replaced by FAIR? See also: Meehl on "Cliometric metatheory".
}

Discuss relationship between FAIR and prose/author's voice.

\reviewerid{unnamed-chunk-32}{


These considerations lead me to some recommendations for strengthening the manuscript. The first is to clarify the relationship between FAIR theory and traditional paper-writing. Is the former meant to replace or enhance the latter? It seems to me that using FAIR to enhance the environment created by traditional theoretical papers is much the better option.  That is, everything that currently exists about the S-R hypothesis could continue to exist, and be supplemented by a FAIR framework outlining the theory and its implications. Development of the FAIR model could exist almost independently of the individual papers, analyses, reviews and meta-analyses (see below). Someone could contribute to the literature on a subject by developing a FAIR framework, much as a good review paper does.
}

Yes, this is absolutely how we intended it. Clarify this.

\reviewerid{unnamed-chunk-33}{


Second, and speaking of Meehl, many of the ideas incorporated here have a relationship to construct validity.  Many of our "theories" of behavior take the form of hypothetical constructs, like intelligence or for that matter the S-R interaction. Much of what is described here seems like a formalization of Meehl's idea of a nomological net-- a network of theoretical connections among abstract entities and empirical observations.
}

This is not exactly true; while that's definitely in line with \emph{my} usage of FAIR theory, other uses are possible.

\reviewerid{unnamed-chunk-34}{


Finally, and this is the only concern about the paper as a publishable manuscript, the last third of the paper where the authors get into the technical aspects of how one would go about building a FAIR theory, is difficult to read. This is where the manuscript reads more like a software manual than a perspectives paper. It would seem out of place in this journal, and frankly it left me discouraged about the prospect of trying this on my own. In its place, I would recommend working an example for an actual theory, with as much of the specific detail as possible removed from the main text.
}

\subsection{Reviewer: 3}\label{reviewer-3}

\reviewerid{unnamed-chunk-35}{


I have read through the manuscript on FAIR theories in psychology. I am extremely sympathetic with the arguments in the paper and believe it will make a valuable contribution.

Drawing on previous work on FAIR principles, the authors of this manuscript argue that psychological theories should be findable, accessible, interoperable, and reusable. These principles are clearly part of a broad movement toward improved and open science in disciplines like psychology.

The authors point out that, despite many reforms since the identification of the "replication crisis" meant to improve scientific rigor (e.g. per-registration), the problems lie with deeper root causes, including a problem with theory development. I could not agree more. For example, there is often little connection between theory and hypotheses. In addition, theories are often so ambiguous as to be unfalsifiable. The key issue, according to the authors, lies in where we should focus reform efforts. To date, reform efforts have focused on improving deductive methods, while overlooking the process of theory construction and improvement. I completely agree.

In the rest of the paper, the authors lay out specific aspects of their argument (e.g. what is a theory? what is the role of formalization in theory development? what is modular publishing? what role should version control play in theory development? etc.). They then discuss a specific example of how to implement the FAIR principles in theory construction and development.

As should be obvious, I am a fan of this paper.
}

\reviewerid{unnamed-chunk-36}{


That said, I wonder if the paper could be improved with a more probing discussion section. If I had one criticism of the paper, I would say that it lets scientists off the hook too easily. An uncritical reader might conclude that the issue is merely one of creating the right set of norms (e.g. version control through Git and archiving through Zenodo). I am doubtful. The sciences (especially physics) have not run into the same problems as psychology and yet they have not always nor always currently practice anything like these FAIR practices. Sure, physics tends to use formal theory, which limits some of the most egregious issues associated with the replication crisis. But that is not true of most theory in biology --- and yet biology has done a better of job of good theory development. I wonder if the authors would be willing to talk more in the discussion about the role of values? We live in a world that is dominated by the logic of incentives: If we just get the incentives right, then we can solve all of our problem. Perhaps. This logic suggests scientists are motivated by mere self-interest (why else would we need the incentives?) and operate in a world of ignorance (e.g. scientists are not aware that their theories are vague and why that is bad for science). I accept that both of these are true, to a large degree. If so, I worry that creating new incentives may only exacerbate problems. It seems to me that we need a change in values. For example, scientists should be embarrassed by publishing vague and ambiguous theories---at least after that has been pointed out to them. I am not trying to be a scold here, but I wonder if there's value in talking about how we might better cultivate norms and values in scientists without requiring more procedures and incentives. After all, why have the problems in the social sciences not so badly infected the sciences? The authors address these issues to some degree (e.g. they mention that they are often asked "who owns a theory?" during seminars. This is precisely the problem! That that question would be asked suggests a deeper problem than incentives---the problem lies in not really understanding what science is or how it works.

Sorry for the rant! I really loved the paper and think it is fine as is. I would love to see a bit more of a discussion about issues like values, but I don't need to see that.
}

Not sure what to do with this! I'm not 100\% convinced that the disciplinary differences as the Reviewer describes exist, and I personally do not take a values based approach - I am more interested in ``how to'' improve social scientific research, and once I know the answer, I am intrinsically motivated to adopt that procedure. I guess I believe that most scholars are doing what they think is best given what they know, and I think that by exposing them to new and better ideas, they will change their ways. I don't see a need to moralize.

\clearpage

\section*{References}\label{references}
\addcontentsline{toc}{section}{References}

\phantomsection\label{refs}
\begin{CSLReferences}{0}{1}
\end{CSLReferences}


\end{document}\grid
